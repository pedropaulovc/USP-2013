\documentclass[brazil, a4paper]{article}
\usepackage{graphicx}
\usepackage[T1]{fontenc}
\usepackage[utf8]{inputenc}
\usepackage{lmodern}
\usepackage{babel}
\usepackage{url}


\begin{document}


\title{Escolha do Projeto}
\author{Pedro Paulo Vezzá Campos \hfill 7538743\\
		Camila Fernandez Achutti \hfill 6795610}
\date{\today}

\maketitle

%\begin{abstract}
%\end{abstract}


\section{Proposta escolhida}
O grupo optou pela proposta 2, animação de algoritmos de aprendizagem
computacional. O algoritmo a ser implementado e simulado visualmente é o
Perceptron.

\section{Sobre o problema de aprendizagem}
Como explicado no enunciado o Perceptron é um algoritmo de classificação 
supervisionada. Sua principal característica e limitação é que apenas possui 
resultados satisfatórios para a classificação de conjuntos linearmente
separáveis.

Assim, instâncias interessantes para o classificador são conjuntos de dados que
podem ou não ser classificados utilizando uma reta para dividir as categorias
de elementos. O caso clássico de conjunto não linearmente separável é a
função ou-exclusivo (XOR). Dentro dos problemas linearmente separáveis,
é interessante considerar casos em que os pontos de dados estão muito ou
pouco separados para verificar a eficiência do classificador.

\section{Cronograma}
Na tabela a seguir estão listadas as tarefas a serem cumpridas para o projeto
de MAC0460 de 2013:

\begin{table}
    \begin{tabular}{|p{5cm}|c|c|c|c|}
    \hline
    Atividade                                                       & setembro & outubro & novembro & dezembro \\ \hline
    Implementar o Perceptron                                        & X        & ~       & ~        & ~        \\
    Buscar na Internet e testar entradas interessantes como exemplo & X        & X       & ~        & ~        \\
    Implementar a interface gráfica                                 & X        & X       & X        & ~        \\
    Escrever o relatório parcial                                    & ~        & X       & ~        & ~        \\ 
    Escrever o relatório final                                      & ~        & ~       & X        & X        \\ \hline
    \end{tabular}
    \caption{Cronograma das tarefas a serem realizadas no semestre}
\end{table}


\end{document}

