\documentclass{article}
\usepackage[brazilian]{babel}
\usepackage[a4paper]{geometry}
\usepackage[T1]{fontenc}
\usepackage[utf8]{inputenc}
\usepackage{lmodern}
%\usepackage{url}
%\usepackage{graphicx}
%\usepackage[pdfborder={0 0 0}]{hyperref}
%\usepackage{amssymb}
%\usepackage{amsmath}

\begin{document}


\title{Avaliação de TCCs Anteriores}
\author{Pedro Paulo Vezzá Campos - 7538743}
\date{\today}
\maketitle

\section{``Escalonamento de aplicações utilizando análise de padrões de uso no InteGrade''}
\subsection{Dados do Trabalho Analisado}
	\begin{description}
		\item[Título] Escalonamento de aplicações utilizando análise de padrões de uso no InteGrade
		\item[Ano] 2008
		\item[Aluno] Thiago Henrique Coraini 
		\item[Orientador] Marcelo Finger
		\item[Nota na Disciplina] $9.5$
	\end{description}

\subsection{Resumo da Monografia}
	O trabalho propõe a implementação de diversos algoritmos de escalonamento em um software de gerenciamento de grades computacionais oportunistas, aquelas que fazem uso de recursos ociosos, o \emph{InteGrade}. Antes das tarefas realizadas no TCC o \emph{InteGrade} já contava com um módulo, o \emph{Local Usage Pattern Analyzer (LUPA)}, para analisar o comportamento de um nó da grade localmente a fim de prever o nível de uso de recursos (CPU e memória) em um período de tempo futuro. Ainda, o projeto contava apenas com um algoritmo de escalonamento de tarefas dentro da grade bastante simples. Este algoritmo apenas excluía da lista de candidatos de nós a serem utilizados os que não atendiam aos requisitos da aplicação no momento da submissão. Os recursos de CPU e memória disponíveis em cada nó não eram considerados pelo algoritmo, subutilizando o \emph{LUPA}. O objetivo principal do TCC foi de integrar o \emph{LUPA} ao \emph{InteGrade}, utilizando as informações fornecidas por ele na lógica de escalonamento. Outra meta do trabalho foi implementar alguns algoritmos de escalonamento e realizar experimentos, tentando descobrir se algum teria um desempenho melhor que outro ao executar alguma aplicação.
	
	Para cumprir estes objetivos, primeiramente o escalonador foi refatorado, o que permitiu que vários algoritmos pudessem ser desenvolvidos sem que um afetasse o outro. Em seguida, foram desenvolvidos quatro algoritmos para o projeto: CanRunGridApplication, que impede a execução em computadores que não manterão ociosidade pelo tempo mínimo definido,  HowLongCanRunGridApplication, que escalona prioritariamente aquele computador que se mantiver disponível por mais tempo, GreedyAverageResourceUsage, que prioriza o computador que tiver mais recursos disponíveis e BestFitAverageResourceUsage, que prioriza computadores que tenham a menor disponibilidade ainda dentro dos requisitos. Em seguida foi descrito o experimento realizado com 20 computadores na Rede Linux comparando os desempenhos dos escalonadores. Não houve diferença expressiva entre um e outro algoritmo.
	
	
\subsection{Avaliação da parte técnica}
	O texto em si é bastante acessível, com uma organização que permite uma leitura sem dificuldades. Não há erros ortográficos ou gramaticais expressivos. O autor traz uma introdução rápida ao contexto no qual o TCC está inserido, permitindo ao leitor rapidamente compreender os objetivos e o desenvolvimento do trabalho.
	
	O foco principal da monografia, o desenvolvimento do escalonador e algoritmos relacionados do \emph{InteGrade} foi uma escolha bastante acertada como tema. Foi possível perceber que o trabalho envolveu estudos amplos tanto dos fundamentos tecnológicos nos quais o \emph{InteGrade} está baseado, \emph{CORBA}, Lua e C++, por exemplo, quanto da teoria de escalonamento implementando alguns algoritmos clássicos na área. O código produzido é relevante uma vez que trouxe benefícios ao software, integrando um componente pouco utilizado e tornando a arquitetura do escalonador mais adaptável para mudanças futuras.
	
	A única crítica ao trabalho está nos experimentos realizados. A escolha da Rede Linux como local para testar as modificações foi um problema já que devido a dúvidas sobre a interferência do \emph{InteGrade} no desempenho da rede o experimento teve de ser interrompido e depois reiniciado, prejudicando confiabilidade dos resultados coletados. Ainda, não houve em nenhum momento uma análise do comportamento dos algoritmos de escalonamento, algo que indicasse que estavam funcionando corretamente.
	
	
\subsection{Avaliação da parte subjetiva}
	Na parte subjetiva o autor faz bom um balanço do seu envolvimento com o projeto \emph{InteGrade}, do qual foi aluno de iniciação científica, expondo os desafios enfrentados, como por exemplo o fato de o software ainda não estar maduro o suficiente e, assim, propenso a bugs diversos. O autor mostrou que fez uma reflexão ao concluir que provavelmente não iria continuar no projeto ao fim de sua IC, pela vontade de conduzir um projeto que fosse exclusivamente dele. O BCC também foi alvo de análise, tanto ao apresentar as disciplinas mais importantes para a sua graduação quanto ao fazer um balanço com um certo ar melancólico da graduação, mas esperançoso com os trabalhos na pós-graduação. 
	
\subsection{Comentários}
	Como dito anteriormente, o texto é de leitura simples, com boa organização e bem redigido. Os trabalhos desenvolvidos foram relevantes o suficente para justificar um trabalho de conclusão de curso. Apenas a parte experimental poderia ser melhor trabalhada. A obra como um todo receberia uma nota $9.5$ pela dedicação ao projeto.

\newpage

\section{``Previsão de Utilização de Recursos por Aplicações no InteGrade''}
\subsection{Dados do Trabalho Analisado}
	\begin{description}
		\item[Título] Previsão de Utilização de Recursos por Aplicações no InteGrade
		\item[Ano] 2009
		\item[Aluno] Fabio Augusto Firmo
		\item[Orientador] Marcelo Finger
		\item[Nota na Disciplina] 9.5
	\end{description}

\subsection{Resumo da Monografia}
	O trabalho propõe uma melhoria em um software de gerenciamento de grades computacionais oportunistas, que aproveitam capacidade computacional ociosa, o \emph{InteGrade}. O problema a ser solucionado é que o programa não é capaz de realizar uma previsão do tempo de processamento e consumo de memória de uma aplicação a ser processada na grade. Antes dos trabalhos desenvolvidos no TCC, o criador da aplicação é que fornecia tais valores, baseado em suposições ou conhecimento prévio de dados. Havia a opção, também, de não fornecer previsão nenhuma. A ausência ou estimativa ruim de valores compromete o desempenho da grade. No caso de estimativas ruins, dois problemas podem acontecer: Estimativas muito conservadoras subutilizam a capacidade computacional disponível enquanto que se estas forem muito agressivas pode haver interferência em outros usos dos computadores que compõem a grade, descumprindo o objetivo de ser uma grade oportunista.

	A solução proposta é baseada em dois aspectos: Primeiro, a definição e busca por execuções anteriores de tarefas similares como parâmetro para a previsão para a tarefa atual. Segundo, a definição de uma fórmula que possa mapear um histórico de execuções na previsão propriamente dita. O trabalho estudou opções estatisticamente simples para modelar os problemas, visando uma implementação igualmente simples. Como critério de similaridade foram escolhidas execuções da mesma aplicação e como critério para consumo computacional a mediana foi a melhor classificada. Comparando os resultados com a bibliografia, os resultados encontram-se em um patamar intermediário. O código relativo ao TCC foi adicionado em um \emph{branch} do desenvolvimento do \emph{InteGrade}.
	
\subsection{Avaliação da parte técnica}
	A redação do trabalho é simples e clara. Não há grandes digressões o que contribui para que ele seja bem enxuto, com 31 páginas no total. Encontrei poucos erros ortográficos, menos de 5, frutos de erros de digitação. A organização do trabalho ficou um pouco confusa, a seção 3.1 -- ``Arquitetura do InteGrade'' poderia vir antes do capítulo 2 -- ``Planejamento'' como forma de fornecer as informações necessárias ao restante do texto. No entanto, os conceitos são suficientemente claros para garantir que a leitura ocorra sem problemas.

	Em vários momentos o autor, ao se deparar com uma decisão de projeto, acabou por optar pela solução mais simples. Vale ressaltar que deve ter sido necessário um tempo considerável para ele se adaptar com a tecnologia que utilizou, \emph{CORBA} e a arquitetura do \emph{InteGrade}, o que deve ser levado em conta na análise dos resultados obtidos. De fato, escolhas mais simples podem ter garantido que o trabalho fosse concluído a tempo já que foi comentado que uma das dificuldades no TCC foi o tempo, por outro lado, não foi possível perceber em sua monografia algum ponto no qual ele tenha sido mais audacioso.
	
	A seção de testes do software é a única mais fraca do trabalho. Foram desenvolvidos alguns testes unitários mas não foram produzidos testes de aceitação ou regressão, de acordo com o autor, pela falta de tempo. Isso é uma deficiência do trabalho pois software mal testado pode causar diversos problemas futuros.
	
\subsection{Avaliação da parte subjetiva}
	Foi feito um balanço breve do curso e foram apresentadas as disciplinas mais úteis na opinião do autor. O balanço poderia ser mais detalhado, o aluno resumiu quatro anos de bacharelado em três parágrafos apenas. Ainda, foi apresentado um pouco do trabalho de bastidores na confecção do TCC, algo muito útil para dar uma perspectiva ao leitor das dificuldades enfrentadas no desenvolvimento da monografia, esta ideia parece bastante útil de ser incorporada à minha monografia. 
	
\subsection{Comentários}
	No geral simpatizo com o trabalho por ser objetivo e sincero. Os resultados são intermediários aos da bibliografia consultada, o que é um ótimo resultado para um TCC. Particularmente concederia uma nota $9.0$, com os únicos descontos sendo a deficiência nos testes do software e a falta de alguma parte mais audaciosa no trabalho.


%\nocite{*}
%\bibliographystyle{abnt-num}
%\bibliography{bibliografia}

\end{document}
