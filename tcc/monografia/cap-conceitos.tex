%% ------------------------------------------------------------------------- %%
\chapter{Conceitos}
\label{cap:conceitos}

\section{Computação em Nuvem}
\label{sec:computacao_nuvem}
Dentro desse contexto surgiu o conceito de computação em nuvem. Apesar de ainda haver controvérsias na sua definição precisa, esse paradigma preza pela noção de utilizar recursos computacionais como poder de processamento, armazenamento e comunicação como um serviço. O pagamento pelos serviços é feito apenas considerando o que foi utilizado, tal como acontece com o fornecimento de energia elétrica. O objetivo final da computação em nuvem é concentrar dados para prover um serviço ubíquo ao usuário, empresas ou pessoas físicas, que delegariam a gestão dessa informação a terceiros competentes para prover um serviço de qualidade e seguro. Grandes empresas da área de tecnologia possuem soluções de computação em nuvem, dentre as quais podemos citar Amazon, Google, Microsoft, IBM, EMC, etc.

Uma importante vantagem da computação em nuvem é que com essa concentração de dados e serviços é possível desenvolver técnicas de otimização do uso de grandes data centers. Segundo estudo realizado por Barroso e Hölzle \cite{barroso:case_energy_proportional} em 5000 servidores do Google, raramente eles permanecem completamente ociosos e dificilmente operam próximos da sua utilização máxima. Na maior parte do tempo estão trabalhando entre 10\% e 50\% do nível máximo. Os autores mostram que justamente nessa faixa de utilização tais servidores são menos eficientes energeticamente. Computação em nuvem é uma candidata a ajudar a melhorar essa perspectiva. Através de virtualização e reposicionamento automático de máquinas virtuais no data center de maneira transparente, uma realidade em produtos pagos ou livres, é possível dimensionar qual parcela do data center estará ativa em um dado momento dependendo da demanda. Vários servidores com pouca utilização poderiam ser, por exemplo, virtualizados em um único servidor físico de modo que este trabalhe com uma utilização que seja mais eficiente.


\section{Escalonamento de Tarefas}
\label{sec:escalonamento_tarefas}

\section{Consumo Energético}
\label{sec:consumo_energetico}


\section{Ambiente de Simulação}
\label{sec:ambiente_simulacao}


