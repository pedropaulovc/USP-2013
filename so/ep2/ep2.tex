\documentclass[brazil, a4paper]{scrartcl}
\usepackage{graphicx}
\usepackage[T1]{fontenc}
\usepackage[utf8]{inputenc}
\usepackage{lmodern}
\usepackage{babel}
\usepackage{url}
\usepackage[usenames,svgnames,dvipsnames]{xcolor}
\usepackage{listings}                   % para formatar código-fonte (ex. em Java)

\lstdefinestyle{customc}{
  belowcaptionskip=1\baselineskip,
  breaklines=true,
  frame=L,
  xleftmargin=\parindent,
  language=C,
  showstringspaces=false,
  basicstyle=\footnotesize\ttfamily,
  keywordstyle=\bfseries\color{green!40!black},
  commentstyle=\itshape\color{purple!40!black},
  identifierstyle=\color{blue},
  stringstyle=\color{orange},
}

\lstdefinelanguage{diff}{
  morecomment=[f][\color{blue}]{@@},     % group identifier
  morecomment=[f][\color{red}]<,         % deleted lines 
  morecomment=[f][\color{ForestGreen}]>,       % added lines
  morecomment=[f][\color{magenta}]{---}, % Diff header lines (must appear after +,-)
  morecomment=[f][\color{magenta}]{+++},
}

\lstset{ %
language=C,                  % choose the language of the code
basicstyle=\footnotesize,       % the size of the fonts that are used for the code
numbers=left,                   % where to put the line-numbers
numberstyle=\footnotesize,      % the size of the fonts that are used for the line-numbers
stepnumber=1,                   % the step between two line-numbers. If it's 1 each line will be numbered
numbersep=5pt,                  % how far the line-numbers are from the code
showspaces=false,               % show spaces adding particular underscores
showstringspaces=false,         % underline spaces within strings
showtabs=false,                 % show tabs within strings adding particular underscores
frame=single,	                % adds a frame around the code
framerule=0.6pt,
tabsize=2,	                    % sets default tabsize to 2 spaces
captionpos=b,                   % sets the caption-position to bottom
breaklines=true,                % sets automatic line breaking
breakatwhitespace=false,        % sets if automatic breaks should only happen at whitespace
escapeinside={\%*}{*)},         % if you want to add a comment within your code
backgroundcolor=\color[rgb]{1.0,1.0,1.0}, % choose the background color.
rulecolor=\color[rgb]{0.8,0.8,0.8},
extendedchars=true,
xleftmargin=10pt,
xrightmargin=10pt,
framexleftmargin=10pt,
framexrightmargin=10pt
}

\begin{document}


\title{Exercício Programa 2 -- Relatório}
\subtitle{MAC0422 -- Sistemas Operacionais}
\author{André Spanguero Kanayama \hfill 7156873\\
		Pedro Paulo Vezzá Campos \hfill 7538743}
\date{\today}

\maketitle

%\begin{abstract}
%\end{abstract}

\section{Enunciado}
Para este segundo exercício-programa de MAC0422 -- Sistemas Operacionais, o 
professor requisitou que os alunos alterassem o \emph{Process Manager} (PM)
para implementar uma biblioteca de semáforos. 


\section{Desenvolvimento da implementação}
Após consultas a tutoriais na Internet, encontramos os principais arquivos
necessários para a implementação de uma biblioteca C que se comunique com o PM:

\begin{description}
	\item[/usr/src/servers/pm/semaphore.c] Local da implementação efetiva dos
	semáforos no PM
	\item[/usr/src/lib/libc/posix/\_semaphore.c] Local do código da biblioteca
	de usuário responsável por abstrair as chamadas de sistema feitas através
	de mensagens no Minix.
	\item[/usr/src/include/minix/callnr.h] Arquivo de macros para os códigos
	definidos para cada uma das chamadas de sistema do Minix.
	\item[/usr/src/servers/pm/table.c] Define a tabela que mapeia um código de
	chamada de sistema definido no arquivo \texttt{callnr.h} para uma função a
	ser executada pelo PM.
	\item[/usr/src/servers/pm/proto.h] Arquivo de protótipos das funções do PM.
\end{description}

\subsection{Código da biblioteca}
O código da biblioteca de usuário é bastante simples. Se resume a montar uma
mensagem a ser enviada para o PM contendo os parâmetros necessários, realizar
a chamada à função \texttt{\_syscall()} e aguardar uma mensagem de retorno 
contendo o resultado da chamada.

É neste ponto que o processo de usuário pode ser terminado e um código de erro
-1 é enviado no caso de um usuário tentar criar um semáforo no momento que o
sistema está com todos os 128 semáforos alocados. Isto foi implementado
na função \texttt{get\_sem()} analisando o código de retorno da chamada de sistema.
Em caso de erro a biblioteca chama a função \texttt{exit()}.

É importante ressaltar que o bloqueio dos processos que fizerem uma operação
``P'' em um semáforo que está com valor atual 0 é feita através do sistema de
troca de mensagens do Minix. O bloqueio acontece na chamada à função
\texttt{\_syscall()} dentro da função  \texttt{p\_sem()} 

\begin{lstlisting}[style=customc]
#include <sys/cdefs.h>
#include "../other/namespace.h"
#include <lib.h>
#include <unistd.h>
#include <stdlib.h>
#include <stdio.h>
#include <minix/callnr.h>

PUBLIC int get_sem (int valor) {
   message m;
   int result;
   m.m1_i1= valor;
   result = _syscall (PM_PROC_NR, GETSEM, &m);
   if(result == 0){
      exit(-1);
   }
   return result;
}

PUBLIC int v_sem (int indice_sem) {
   message m;
   m.m1_i1 = indice_sem;
   return (_syscall (PM_PROC_NR, VSEM, &m));
}

PUBLIC int p_sem (int indice_sem) {
   message m;
   m.m1_i1 = indice_sem;
   return (_syscall (PM_PROC_NR, PSEM, &m));
}

PUBLIC int free_sem (int indice_sem) {
   message m;
   m.m1_i1 = indice_sem;
   return (_syscall (PM_PROC_NR, FREESEM, &m));
}
\end{lstlisting}

\subsection{Modelagem de um semáforo}
O arquivo \texttt{/usr/src/servers/pm/semaphore.c} contém a implementação
verdadeira das funções requisitadas no enunciado.

Um semáforo foi modelado como uma \texttt{struct} C com a seguinte definção:
\begin{lstlisting}[style=customc]
struct semaforo {
   int id, valor; 
   int fila_pid[TAM_FILA]; 
   int begin;
   int end; 
   int pcount;
   int dono;
} semaforos[NR_SEMS];
\end{lstlisting}

A macro \texttt{NR\_SEMS} foi definida como 128, conforme enunciado.
A utilidade de cada componente está descrita a seguir:

\begin{description}
	\item[id] É uma ID criada aleatoriamente pela biblioteca e na faixa $[1, 1000000]$
	\item[valor] Indica quantos processos podem entrar na região crítica no momento
	\item[fila\_pid] É uma fila implementada usando um vetor C que indica quais
		processos (Representados pelas suas posições na tabela de processos do
		PM) estão bloqueados aguardando liberação para entrar na região
		crítica.
	\item[begin, end] Apontam respectivamente o início e o fim da fila no vetor
		\texttt{fila\_pid}.
	\item[pcount] Número de processos que estão aguardando liberação para entrar
		na região crítica.
	\item[dono] Posição na tabela de processos do processo criador do semáforo.
\end{description}

\subsection{Implementação das operações de manipulação de um semáforo}
O arquivo \texttt{semaphore.c} contém as seguintes funções implementadas:

\begin{description}
	\item[void init\_sems (void)] Invocada na primeira vez que é requisitado um
		semáforo.
	\item[void desaloca\_pid (int who\_p)] Recebe como parâmetro a posição na tabela
		de processos de um processo	que terminou a sua execução. Varre toda a
		tabela de semáforos e desaloca qualquer semáforo que tenha sido alocado
		a este processo.
	\item[int sem\_exists (int sem)] Varre a tabela de semáforos e retorna 1 caso
		a ID passada esteja presente e 0 caso contrário.
	\item[int authorized (int sem\_index)] Dada uma posição na tabela de semáforos,
		retorna 1 caso o processo que fez a requisição à biblioteca ou algum de
		seus ancestrais seja dono do semáforo referenciado.
	\item[int do\_getsem (void)] Tenta criar um novo semáforo para o processo
		chamador. Retorna a ID do novo semáforo na mensagem de retorno ou -1
		em caso de erro. O código que é responsável por terminar um processo
		que requisitou um semáforo quando todos já estavam alocados é
		responsabilidade da biblioteca de usuário, como explicado anteriormente.
	\item[int do\_vsem (void)] Aplica a operação ``V'' no semáforo passado como
		parâmetro via o primeiro campo da mensagem do Minix. Caso haja algum
		processo bloqueado, libera-o, caso contrário, incrementa o valor atual
		do semáforo. Retorna -1 via mensagem caso a ID do semáforo seja inválida ou o usuário
		não esteja autorizado de acordo com a função \texttt{authorized()}.
	\item[int do\_psem (void)] Aplica a operação ``P'' no semáforo passado como
		parâmetro via o primeiro campo da mensagem do Minix. Caso o valor atual
		do semáforo seja 0, bloqueia-o, caso contrário, decrementa o valor atual
		do semáforo e libera o processo. Retorna -1 via mensagem caso a ID do semáforo seja
		inválida ou o usuário não esteja autorizado de acordo com a função \texttt{authorized()}.
	\item[int do\_freesem (void)] Libera o semáforo passado como parâmetro no
		primeiro campo da mensagem do Minix apenas se o remetente da
		mensagem for o dono do semáforo. Retorna 0 em caso de sucesso e -1
		caso o rementente não esteja autorizado a realizar a operação ou a ID do
		semáforo não seja válida.
\end{description}

\subsection{Desalocação de semáforos após o fim de um processo}
Para satisfazer à restrição de desalocar todos os semáforos de um processo quando
ele terminar a sua execução, vasculhamos o código fonte do PM até encontrar a
função \texttt{exit\_proc()} dentro do arquivo
\texttt{/usr/src/servers/pm/forkexit.c}. Esta função é responsável por desalocar
todas as estruturas utilizadas pelo PM no controle da execução de um processo.

A edição feita foi incluir mais uma linha nesta função. Nela, é feita uma
chamada à função \texttt{desaloca\_pid()}, responsável por desalocar todos os
semáforos de um processo, como descrito anteriormente. O \texttt{diff} resultante
é:

\begin{lstlisting}[language=diff]
  if (dump_core && (rmp->mp_flags & PRIV_PROC))
	dump_core = FALSE;

  proc_nr = (int) (rmp - mproc);	/* get process slot number */
  proc_nr_e = rmp->mp_endpoint;
> /*??????????????????????????????????????????????????*/
> /*??????????????????????????????????????????????????*/
>   desaloca_pid(proc_nr);
> /*??????????????????????????????????????????????????*/
> /*??????????????????????????????????????????????????*/

  /* Remember a session leader's process group. */
  procgrp = (rmp->mp_pid == mp->mp_procgrp) ? mp->mp_procgrp : 0;
\end{lstlisting}

\section{Dificuldades enfrentadas}
Perdemos bastante tempo tentando entender o porquê de mesmo após alterarmos
e recompilarmos o código da biblioteca de usuário, o comportamento do nosso
programa de teste não mudava para o esperado. Concluímos que a ``culpada'' foi
a ligação estática de bibliotecas ao binário de teste compilado. A implementação
da biblioteca de usuário é copiada para o binário final apenas durante a
compilação do binário do usuário. Quando recompilamos nosso código de teste
o comportamento passou a ser o esperado.

Na implementação da biblioteca de semáforos, vimos o PM utilizar a variável
global \texttt{who\_p} para referenciar o remetente da mensagem que está sendo
processada. Acreditamos que essa variável fosse o PID do processo chamador mas
na verdade \texttt{who\_p} é a posição na tabela de processos referente ao
remetente.


\appendix 
\section{Implementação final da biblioteca}

\begin{lstlisting}[style=customc]
/************************************
 * 
 * semaphore.c
 * 
 ************************************/ 
#include "pm.h"
#include "param.h"
#include "glo.h"
#include "mproc.h"
#include <sys/wait.h>
#include <assert.h>
#include <signal.h>
#include <stdio.h>
#include <stdlib.h>
#include <string.h>
#include <sys/types.h>
#include <signal.h>

/* Numero maximo de semaforos */
#define NR_SEMS      128        
#define TAM_FILA     1000       
#define NO_SEM      -1          
#define INIT_SEMVAL  1000       
#define MAX_SEMVAL   1000000    
#define INT_LIMIT    0x7fffffff 
#define ESEMVAL      0x8000000
/*Ultrapassou o limite de semaforos*/  
#define EFULL        0xDEADDEAD
#define NADA   0
/*Erro, retorna 0*/          
#define RETERR       -1          


FORWARD _PROTOTYPE( int sem_exists, (int sem)                            );
FORWARD _PROTOTYPE( void init_sems, (void)                              );

/*Definicao da estrutura do semaforo*/
struct semaforo {
   int id, valor; 
   int fila_pid[TAM_FILA]; 
   int begin;
   int end; 
   int pcount;
   int dono;
} semaforos[NR_SEMS];
int qtd_semaforos_ativos = 0;

/*Funcao executada uma vez, quando ainda nao existe nenhum semaforo*/
PRIVATE void init_sems (void) {
   int i;

   for (i = 0; i < NR_SEMS; i++) {
      semaforos[i].id = NO_SEM;
      semaforos[i].valor = 0;
      semaforos[i].pcount = 0; 
      semaforos[i].fila_pid[0] = NADA;  
      semaforos[i].begin = 0;
      semaforos[i].end = 0;
   }
   qtd_semaforos_ativos = 0;
}

/*Funcao usada pelo PM, para que quando um processo morra, todos os seus semaforos sejam liberados*/
PUBLIC void desaloca_pid (who_p)
    int who_p;
{
    int i;
    for (i = 0; i < NR_SEMS; i++) {
      if (semaforos[i].dono == who_p){
          qtd_semaforos_ativos--;

          semaforos[i].id = NO_SEM;
          semaforos[i].valor = 0;

          semaforos[i].pcount = 0;
          semaforos[i].fila_pid[0] = NADA;
          semaforos[i].begin = 0;
          semaforos[i].pcount = 0;

      }
   }

    
}

/*Funcao que verifica se um semaforo ja existe*/
PRIVATE int sem_exists (sem)
   int sem;
{
   int i;
   if (sem <= 0 || sem > INT_LIMIT - 1) return 0;

   if (qtd_semaforos_ativos < 1) return 0;
   for (i = 0; i < NR_SEMS; i++) {
      /*Se encontrar uma id igual a passada, retorna 1*/
      if (semaforos[i].id == sem) return 1;  
   }
   return 0;
}

/*Funcao que verifica se o processo pode fazer uma operacao com o semaforo*/
PRIVATE int authorized (sem_index) 
   int sem_index;
{
   int i;
   int mproc_index = -1;
   int pid_atual = who_p;
   register struct mproc *atual;
   atual = &mproc[pid_atual];
  
   while(atual->mp_parent != pid_atual && pid_atual > 0 && semaforos[sem_index].dono != pid_atual){
      /*Verifica se os processos "pais" sao donos do semaforo, e se um deles for, o processo eh autorizado*/
      pid_atual = atual->mp_parent;
      atual = &mproc[pid_atual];
      
   }

   if(semaforos[sem_index].dono == pid_atual)
      return 1;
   return 0;
}

/*************************************************GET SEM****************************************************************/
PUBLIC int do_getsem (void) {
   int resultado = RETERR;
   int id_semaforo;
   /*mensagem*/
   int valor = m_in.m1_i1;
   int i;

   if (qtd_semaforos_ativos == 0) init_sems ();


   if (qtd_semaforos_ativos >= NR_SEMS) {
      m_in.m_type = EFULL;
      return RETERR;
   }

   for (i = 0; i < NR_SEMS && resultado == RETERR; ++i) {
      if (semaforos[i].id == NO_SEM) {
         /*Gera um id aleatorio para o semaforo*/
         id_semaforo = rand() % MAX_SEMVAL + 1;
         /*Se semaforo ja existe, soma 100 no id ateh achar um numero que nao exista*/ 
         while (sem_exists (id_semaforo)) id_semaforo += 100;
         qtd_semaforos_ativos++;  
         resultado = id_semaforo;  
          
         semaforos[i].id = id_semaforo;  
         semaforos[i].valor = valor;  
          
         semaforos[i].pcount = 0;  
         semaforos[i].fila_pid[0] = NADA;  
         semaforos[i].begin = 0;
         semaforos[i].end = 0;
         semaforos[i].dono = who_p;
      }
   }
  
   return resultado;
}


/*************************************************V SEM***************************************************/
PUBLIC int do_vsem (void) {
    
   int resultado = RETERR;  
    
   int sem = m_in.m1_i1;

   int i;
   int index;
    
   if (sem <= 0 || sem > INT_LIMIT - 1 || qtd_semaforos_ativos < 1) {
      m_in.m_type = EINVAL;
      return RETERR;
   }
    
   index = -1;
   /*Procura o indice que corresponde ao semaforo no vetor de semaforos*/
   for (i = 0; i < NR_SEMS; i++){
      if(semaforos[i].id == sem){
          index = i;
          break;
      }
   }

   /*Nao achou o semaforo*/
   if(index == -1)
      return RETERR;    

   /*Se nao for dono ou filho do dono nao pode fazer a operacao*/
   if(!authorized(index))
      return RETERR;

   if (semaforos[index].valor == MAX_SEMVAL) {
      m_in.m_type = EBUSY;
      return RETERR;
   }
    
   (semaforos[index].valor)++;
    
   if (semaforos[index].pcount != 0) {
      int next = semaforos[index].begin;
      int proc_nr = semaforos[index].fila_pid[next];
       
      if (proc_nr <= NADA || proc_nr >= NR_PROCS) return RETERR;
      
       
      if (next == semaforos[index].end) {
         semaforos[index].fila_pid[next] = NADA;  
         semaforos[index].begin = 0;  
         semaforos[index].end = 0;  
      }else if (next == (TAM_FILA - 1)) {
         semaforos[index].fila_pid[TAM_FILA - 1] = NADA;
         semaforos[index].begin = 0;  
      }else {
         semaforos[index].fila_pid[next] = NADA;
         next++;  
         semaforos[index].begin = next;
      }
       
      (semaforos[index].pcount)--;  
      resultado = 0;
       
      setreply (proc_nr, resultado);
   }
   
   return resultado;
}

/******************************************************P SEM**********************************************************/
PUBLIC int do_psem (void) {
    
   int resultado = RETERR;  
    
   int sem = m_in.m1_i1;

   int index;
    
   int i;
    
   if (sem <= 0 || sem > INT_LIMIT - 1 || qtd_semaforos_ativos < 1) {
      m_in.m_type = EINVAL;
      return RETERR;
   }
   
   index = -1;
   /*Procura o indice do semaforo no vetor de semaforos*/
   for (i = 0; i < NR_SEMS; i++){
      if(semaforos[i].id == sem){
          index = i;
          break;
      }
   }
   
   /*Semaforo nao existe, retorna -1*/
   if(index == -1)
      return RETERR;    

   /*Verifica se o proccesso pode fazer a operacao*/
   if(!authorized(index))
      return RETERR;
    
   if (semaforos[index].valor == -MAX_SEMVAL) {
      m_in.m_type = EBUSY;
      return RETERR;
   }
    
   if (semaforos[index].pcount == TAM_FILA) {
      m_in.m_type = EBUSY;
      return RETERR;
   }
   
    
   (semaforos[index].valor)--;
    
    if (semaforos[index].valor < 0) {
      int last = semaforos[index].end;
      int proc_nr = who_p;

      resultado = 0;
      if (semaforos[index].pcount == 0) {
         semaforos[index].fila_pid[last] = proc_nr;
      }else if (last == (TAM_FILA - 1)) {
         last = 0;
         semaforos[index].end = last;
         semaforos[index].fila_pid[last] = proc_nr;
      }else {
         ++last;
         semaforos[index].end = last;
         semaforos[index].fila_pid[last] = proc_nr;
      }
       
      (semaforos[index].pcount)++;  
      
       
      return (SUSPEND);
   }

   return resultado;
}

/**************************************************FREE SEM**************************************************************/
PUBLIC int do_freesem (void) {
    
   int resultado = RETERR;  
    
   int sem = m_in.m1_i1;
   int i;
    
    
   int index;
   
   /*Passou valor errado*/
   if (sem <= 0 || sem > INT_LIMIT - 1) {
      m_in.m_type = EINVAL;
      return RETERR;
   }
   
   /*Procura indice do semaforo no vetor de semaforos*/
   for (i = 0; i < NR_SEMS; i++){
      if(semaforos[i].id == sem){
          index = i;
          break;
      }
   }

   /*Semaforo nao existe*/
   if(index == -1)
      return RETERR;    
   
   /*Verifica se o processo eh dono para poder libera-lo*/ 
   if(semaforos[index].dono != who_p)
      return RETERR;
   
   if (semaforos[index].pcount != 0) {
      m_in.m_type = EBUSY;
      return RETERR;
   }

   resultado = 0; 
   qtd_semaforos_ativos--;
    
   semaforos[index].id = NO_SEM;
   semaforos[index].valor = 0;
    
   semaforos[index].pcount = 0;  
   semaforos[index].fila_pid[0] = NADA;  
   semaforos[index].begin = 0;
   semaforos[index].pcount = 0;
   
   return resultado;
}
\end{lstlisting}


\end{document}

