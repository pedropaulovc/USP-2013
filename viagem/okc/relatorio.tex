\documentclass{article}
\usepackage[brazilian]{babel}
\usepackage[a4paper]{geometry}
\usepackage[T1]{fontenc}
\usepackage[utf8]{inputenc}
\usepackage{lmodern}
\usepackage{rotating}
\usepackage{url}
\usepackage{graphicx}
%\usepackage[pdfborder={0 0 0}]{hyperref}
%\usepackage{amssymb}
%\usepackage{amsmath}

\begin{document}
\title{Relatório de Viagem ao FIE2013}
\author{Pedro Paulo Vezzá Campos - 7538743}
\date{\today}

\maketitle

%TODO: Verificar trajetos

\hfill À Comissão de Graduação do IME/USP:\\

\vspace{1cm}

Este documento é um relatório descrevendo as atividades realizadas durante a minha participação no \emph{2013 Frontiers in Education Conference}, que ocorreu de 22 a 26 de outubro de 2013 em Oklahoma City e que foi financiado com verba do programa Pró-Int. Permaneço à disposição para quaisquer esclarecimentos.

\vfill

\begin{flushright}
	\noindent\rule{8cm}{0.4pt} \\
	Pedro Paulo Vezzá Campos\\
	\url{pedro@vezza.com.br}\\
	(11) 97132-1145\\
\end{flushright}

\newpage

\section{Introdução}
Como parte da minha iniciação científica no Projeto Apoio BCC, produzi um artigo científico juntamente com o aluno Jackson José de Souza e sob orientação do coordenador pedagógico do IME, Giuliano Salcas Olguin. O título do trabalho foi ``A Survey on the Mathematical Emphasis in Brazilian Computer Science Curricula''. O congresso escolhido para a publicação foi o $43^o$ \emph{Frontiers in Education Conference}, realizado entre 22 e 26 de outubro de 2013 em Oklahoma City. O congresso apresenta-se como um local ideal para a confluência de ideias inovadoras na área de educação em engenharia e ciência vindas do mundo todo.

\section{Objetivo}
Meu objetivo com esta viagem era de primeiramente apresentar o trabalho no congresso e paralelamente estabelecer contatos com pesquisadores na área. Por fim, a experiência de uma viagem internacional sozinho é bastante enriquecedora por si mesma.

\section{Descrição das Atividades Realizadas}
A viagem iniciou no dia 22 de outubro com a chegada em Oklahoma City. Neste dia foi feito o \emph{checkin} no hotel e um reconhecimento das redondezas do centro de convenções.

No dia 23 de outubro o evento iniciou com o credenciamento e workshops pagos à parte. Como não comprei nenhum por restrições orçamentárias, aproveitei o dia para conhecer os principais pontos turísticos de Oklahoma City tais como \emph{Bricktown}, \emph{Myriad Gardens}, o Memorial ao Atentado de 1995 e o Museu de Arte de Oklahoma City.

O dia 24 marcou o início das atividades principais do congresso. Havia 4 seções de apresentação de artigos, com 1h30 cada uma. Em cada seção havia normalmente 5 artigos para ser apresentados. Dada a grande quantidade de artigos vista por dia (Aproximadamente 20 por dia) abaixo estão elencados alguns dos trabalhos mais relevantes vistos:

\begin{description}
	\item[\emph{A taxonomy of exercises to support individual learning paths in initial programming learning}] A autora apresenta uma estratégia para classificar diversos problemas de programação de maneira a aprendizado de maneira gradual, respeitando as dificuldades dos alunos. O resultado final é um ``grafo de dificuldade'' com problemas sendo vértices e níveis de dificuldade sendo as arestas. Um sistema de aprendizado automatizado pode apresentar problema de programação em uma sequência personalizada ao aluni.
	\item[\emph{Method for teaching parallelism on heterogeneous many-core processing using research projects}] O autor apresenta sua experiência como professor de uma disciplina de arquitetura de computadores na qual ele aborda o tópico de processamento multi-core através de desafios aos alunos. 
	\item[\emph{PBL in teaching computing: An overview of the last 15 years}] A autora faz uma revisão histórica da técnica de ensino conhecida como ``\emph{Problem based learning}''. Sua apresentação foi focada em apresentar evoluções na metodologia, vantagens e desvantagens e maneiras de aplicar a técnica quando ao lidar com disciplinas de Computação especificamente.
	\item[\emph{iQuiz: Integrated assessment environment to improve Moodle Quiz}] A pesquisa apresentou um plugin ao sistema Moodle de aprendizado que é mais genérico e mais fácil de manipular que o módulo Quiz já embutido no programa. Foi mostrado como o sistema é fácilmente expansível, permitindo, inclusive, criar uma ``federação'' de bancos de perguntas.
\end{description}


No dia 25 de outubro as palestras continuaram como de costume. As que mais me interessaram no dia foram:

\begin{description}
	\item[\emph{A model to support a learning object repository for web-based courses}] Este artigo foi apresentou o componente de banco de dados que foi utilizado no trabalho \emph{iQuiz: Integrated assessment environment to improve Moodle Quiz}. Aqui foram detalhadas as estratégias empregadas na implementação de um BD especializado no armazenamento de \emph{Learning Objects}.
	\item[\emph{Drafting program educational objectives for undergraduate engineering degree programs}] Aqui, o autor foca no assunto de preparar propostas de ``\emph{Program Educational Objectives}'' (PEO) desenhados para cursos de graduação em engenharia. Os PEO podem ser vistos com os equivalentes ao Programa Político-Pedagógico (PPP) das universidades brasileiras. 
	\item[\emph{Physics of computing as an introduction to computer engineering}] Este trabalho apresenta uma metodologia de ensino da área de circuitos e sistemas integrados de uma maneira diferente dos cursos tradicionais, que utilizam uma metodologia construtiva para apresentar os assuntos de projeto lógico e organização de computadores. Em contraste, ele apresenta um curso mais voltado para a área física, evidenciando como determinar propriedades básicas de computadores como velocidade e consumo energético. Por fim, são vistos os principais \emph{trade-offs} envolvidos em projetos lógicos, muitas vezes atrelados a restrições físicas.
	\item[\emph{Carry-on effect in extreme apprenticeship}] Pesquisadores finlandeses desenvolveram uma técnica denominada extreme apprenticeship (XA) para o ensino de programação e posteriormente de matemática em cursos de graduação. Toma medidas radicais como a abolição de aulas no formato tradicional. Ao invés disso, a técnica é baseada em um aprendizado passo-a-passo e incremental feito pelo próprio aluno durante aulas em laboratório. Um pequeno batalhão de monitores, juntamente com o professor é responsável por prestar auxílio, explicações individualizadas e corrigir uma média de 100 exercícios por semestre por aluno.
	\item[\emph{Multiple intelligence approach and competencies applied to Computer Science I}] O autor apresentou técnicas de ensino lúdico que estimulem as ``múltiplas inteligências'' que o ser humano possui (Cinestésica, matemática, sensorial, etc.)
\end{description}

O dia 26 de outubro foi o último dia do FIE. Devido ao horário do meu voo de retorno ao Brasil, apenas participei das atividades da manhã.

\begin{description}
	\item[\emph{The CS2013 computer science curricula guidelines project}] Esta seção especial tratou das mudanças no currículo de referência produzido pela ACM em conjunto com a IEEE. Após a apresentação, formaram-se grupos de discussão mediado por um membro do comitê de elaboração do documento para o compartilhamento de suas experiências. Bastante proveitoso, um relato detalhado foi enviado para os representantes de curso do BCC e professores participantes da comissão de reformulação do currículo do BCC.
	\item[\emph{A survey on the mathematical emphasis in Brazilian computer science curricula}] Este foi o artigo desenvolvido pelo projeto Apoio BCC e apresentado por mim. Aqui é feita uma análise quantitativa do enfoque matemático de 11 cursos de graduação do Brasil. As conclusões foram que a disciplina de matemática está em decadência em termos de currículos de referência e em termos de cursos de graduação brasileiros.
\end{description}

Além das atividades regulares do evento, vale a pena também ressaltar outras atividades realizadas, tais como as visitas ao Museu do Cowboy e da cultura Western e ao National Weather Center em Norman, OK. Ambas visitas trouxeram um enriquecimento ao apresentar assuntos pouco estudados por mim, cultura nativa americana e meteorologia.

\section{Conclusão e Agradecimentos}
Como foi possível ver, a viagem ao FIE2013 foi bastante intensa e enriquecedora. Além de contato com técnicas de ensino e discussões diversas, a viagem foi marcante pela experiência de contato com pesquisadores do mundo todo. Foi possível ver como todos possuíam experiências variadas e complementares. Considero, ainda, que cumpri meus objetivos, retornando a São Paulo com ânimo renovado para a conclusão da minha graduação. 

Presto meus agradecimentos à Comissão de Graduação do IME/USP pela oportunidade concedida e à Pró-reitoria de Graduação pelo financiamento disponibilizado.

%\nocite{*}
%\bibliographystyle{abnt-num}
%\bibliography{bibliografia}

\end{document}
